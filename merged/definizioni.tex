\documentclass[11pt]{article}


\documentclass[11pt]{article}
\usepackage[italian]{babel}
\usepackage[utf8]{inputenc}	% Para caracteres en español
\usepackage{amsmath,amsthm,amsfonts,amssymb,amscd}
\usepackage{multirow,booktabs}
\usepackage[table]{xcolor}
\usepackage{fullpage}
\usepackage{lastpage}
\usepackage{enumitem}
\usepackage{fancyhdr}
\usepackage{mathrsfs}
\usepackage{wrapfig}
\usepackage{graphicx}
\graphicspath{ {./images/} }
\usepackage{setspace}
\usepackage{calc}
\usepackage{multicol}
\usepackage{cancel}
\usepackage[retainorgcmds]{IEEEtrantools}
\usepackage[margin=3cm]{geometry}
\usepackage{amsmath}
\newlength{\tabcont}
\setlength{\parindent}{0.0in}
\setlength{\parskip}{0.05in}
\usepackage{empheq}
\usepackage{framed}
\usepackage[most]{tcolorbox}
\usepackage{xcolor}
\usepackage{FiraSans}
\usepackage{listings}
\usepackage{color} %red, green, blue, yellow, cyan, magenta, black, white
\definecolor{mygreen}{RGB}{28,172,0} % color values Red, Green, Blue
\definecolor{mylilas}{RGB}{170,55,241}
\colorlet{shadecolor}{orange!15}
\parindent 0in
\parskip 12pt
\geometry{margin=1in, headsep=0.25in}
\theoremstyle{definition}
\newtheorem{defn}{Definizione}
\newtheorem{reg}{Regola}
\newtheorem{oss}{Osservazione}
\newtheorem{exer}{Esecizio}
\newtheorem{note}{Nota}
\newtheorem{thm}{Teorema}[section] % reset theorem numbering for each chapter
\theoremstyle{plain}
\newcommand{\restr}[2]{%
\mathchoice{%
\restriction{#1}{#2}{\displaystyle}%
}{%
\restriction{#1}{#2}{\textstyle}%
}{%
\restriction{#1}{#2}{\scriptstyle}%
}{%
\restriction{#1}{#2}{\scriptscriptstyle}%
}%
}

\lstset{language=Matlab,%
    %basicstyle=\color{red},
    breaklines=true,%
    morekeywords={matlab2tikz},
    keywordstyle=\color{blue},%
    morekeywords=[2]{1}, keywordstyle=[2]{\color{black}},
    identifierstyle=\color{black},%
    stringstyle=\color{mylilas},
    commentstyle=\color{mygreen},%
    showstringspaces=false,%without this there will be a symbol in the places where there is a space
    numbers=left,%
    numberstyle={\tiny \color{black}},% size of the numbers
    numbersep=9pt, % this defines how far the numbers are from the text
    emph=[1]{for,end,break},emphstyle=[1]\color{red}, %some words to emphasise
    %emph=[2]{word1,word2}, emphstyle=[2]{style},    
}


\newlength{\totbarheight}
\newlength{\bardepth}



\begin{document}
\defn{}{Una equazione di ordine n è una equazione del tipo:
\[
    F(x,y(x),y'(x),\ldots,y^{(n-1)}(x),y^{(n)}(x))=0
\]
\[
    x \in I \subseteq \mathbb{R}
\]
dove l'incognita è la qualunque y(x). F è funzione di (n+2) variabili $x,y(x),y'(x)\ldots.$
}
\defn{Soluzione (curva) integrale}{La soluzione di una EDO di ordine n sull'intervallo I 
    \begin{equation}\label{eq:soluzione}
         F(x,y(x),y'(x),\ldots) = 0
     \end{equation}
\[
    x \in I \subseteq \mathbb{R}
\]

$\varphi(x)$ che sia definita (almeno) in I e ivi derivabile fino all'ordine n per cui valga \ref{eq:soluzione}, ovvero:
\[
    F(x,\varphi(x),\varphi ' (x), \ldots ) = 0 
\]

$\forall x \in I$

Chiaramente cambia a seconda dell'intervallo
}
\defn{Integrale Generale}{Si chiama integrale \textbf{generale} di \ref{eq:soluzione} in I l'insieme di tutte le soluzioni di \ref{eq:soluzione} in I}
\defn{Forma normale}{Una Equazione Differenziale Ordinaria (EDO) di ordine n si dice in forma normale se è in forma

    \[
        y^{(n)} = f(x,y(x),y'(x), \ldots ,y^{(n-1)}), x \in I
    \]
    
    Esempio:
    \[
        y'''=-5y'+sinx
    \]
    Quella sopra è un EDO di III ordine normale.
}
\defn{EDO di ordine n lineare}{Una EDO di ordine n si dice lineare se è nella forma
    \[
        a_n(x)y^{(n)}(x)+a_{n-1}(x)y^{(n-1)}+ \ldots + a_2(x)y''(x)+a_1(x)y'(x)+a_0y(x)=f(x),x \in I
    \]

    Dove le funzioni \[
        a_0(x),a_1(x),a_2(x), \ldots,a_n(x),f(x)
    \]

    sono assegnate (continue) in I

    Esempio:
    \[
        xy''+5y = sin x
    \]

}
\defn{Funzione continua in più variabili}{ Sia una funzione e sia $P_0$ un punto di accumulazione per a, si dice che la funzione è continua in $P_0$ se:

    \[
        \lim_{ P \to P_0 } f(P) = f(P_0)
    \]

    se $P_0$ è un punto isolato per $A$ per convenzione $f$ è continua

}
\defn{}{Limite per coordinate polari: 

    \[
        \lim_{ \rho \to 0^{+} } f(x_0+\rho cos\theta,y_0+\rho sen \theta) = l
    \]

    ovvero che:

    \[
        \forall \varepsilon>0,\exists \sigma>0
    \]

    per ogni:

    \[
        \underbrace{0<\rho<\sigma}_{\rho \rightarrow 0^{+}},\forall \theta \in (0,2\pi)
    \]

    si ha:

    \[
        |f(x_0+\rho cos\theta, y_0+\rho sin \theta ) -l| < \varepsilon
    \]
}
\defn{Curva}{
Una curva è un'applicazione continua:

\[
    \varphi: I \subset \mathbb{R} \rightarrow \mathbb{R}^{n}
\]

per $I=[a,b]$:

\[
   \bar{\varphi}(t) = (\varphi_1(t),\varphi_2(t), \ldots ,\varphi_n(t)) \in \mathbb{R}^{n}
\]

Le equazioni parametriche sono:

$\varphi=$\begin{equation}
    \begin{cases}
           x_1(t)=\varphi_1(t)\\
           x_2(t) = \varphi_2(t)\\
   &\;\;\vdots \notag \\
           x_n(t) = \varphi_n(t)

    \end{cases}
\end{equation}

}
\defn{Curva regolare}{
Una curva si dice regolare se l'applicazione $\varphi$ è di classe $C^{1}$ (le derivate prime sono continue) e $\varphi'(t) \neq 0$

In particolare $\varphi'(t) \neq 0$ significa che il vettore:

\[
    (\varphi_1'(t), \ldots ,\varphi_n'(t)) = \varphi'(t)
\]

non ha mai tutte le componenti contemporaneamente nulle.
}
\defn{}{ Si dice che $f$ è differenziabile in $\bar{x} \in A $ se $f$ è derivabile in $\bar{x} $ ed inoltre vale la seguente relazione:

    \[
        \lim_{ h \to 0 } \frac{f( \bar{x} + \bar{h} ) - f( \bar{x} ) - \langle  \nabla f(\bar{x} ), \bar{h} \rangle}{|\bar{h} |} =0
    \]


    dove $\bar{h} \in \mathbb{R}^{n}$ e in particolare $|\bar{h} | = \sqrt{\sum^{n}_{i=1} h_i^{2}}$, inoltre:

    \[
    \langle  \nabla f(\bar{x} ), \bar{h} \rangle = \frac{\partial f}{\partial x_1}(\bar{x} )h_1+ \frac{\partial f}{\partial x_2}(\bar{x} ) h_2+ \ldots  + \frac{\partial f}{\partial x_n}(\bar{x} ) h_n
    \]

    Se $f$ è differenziabile in ogni punto di $A$ si dice che $f$ è differenziabile in $A$
}
\defn{}{ Definisco $L : \mathbb{R}^{n} \rightarrow \mathbb{R}$ (funzione lineare) è l'applicazione che ad $\bar{h} $ associa il prodotto scalare:

    \[
        \langle  \nabla f(\bar{x} ) , \bar{h} \rangle
    \]

    si chiama differenziale di $f$ in $\bar{x} $ e si indica con $d f(x)$ è un'applicazione lineare da $\mathbb{R}^{n}$ in $\mathbb{R}$ della variabile $\bar{h} $

    \[
        d f(\bar{x} ) (\bar{h} ) = \langle \nabla f(\bar{x} ) , \bar{h}  \rangle
    \]
}
\defn{}{Sia $A \subseteq \mathbb{R}^{n}$ con $A$ aperto, $f:A \rightarrow R$ una funzione e sia $x_0 \in A$. Si dice che $f$ è differenziabile in $x_0$ se $\exists L:\mathbb{R}^{n}\rightarrow \mathbb{R}$ funzione lineare t.c.:

    \[
        \lim_{ h \to 0 } \frac{f(x_0+h) - f(x_0) -L(h)}{|h|} = 0
    \]

    dove $L(h) = \langle \nabla f(x_0), h \rangle$
}
\defn{Derivate seconde}{ Se $\exists$ sono della forma:

    \[
        \frac{\partial^{2} f}{\partial x_i \partial x_j} = f_{x_i x_j}
    \]

    e si ottengono al variare di $i,j$ da $1$ ad $n$.
}
\defn{Matrice Hessiana}{Attraverso le derivate seconde si ottiene una matrice $n\times n$ che ha come elementi tutte le derivate seconde. Questa è chiamata matrice Hessiana, e si indica come:

\[
    Hf= \begin{bmatrix}
        \frac{\partial^{2} f}{\partial x_1 \partial x_1} & \frac{\partial^{2} f}{\partial x_1 \partial x_2} & \ldots & \frac{\partial^{2} f}{\partial x_1 \partial x_n}\\
        \frac{\partial^{2} f}{\partial x_2 \partial x_1}& \ldots & \ldots & \ldots \\
        \ldots & \ldots & \ldots & \ldots \\
        \frac{\partial^{2} f}{\partial x_n \partial x_1}& \ldots & \ldots & \frac{\partial^{2} f}{\partial x_n \partial x_n} 
    \end{bmatrix} = \begin{bmatrix}
        f_{x_1 x_1}& f_{x_1 x_2}& \ldots & f_{x_1 x_n}\\
        f_{x_2 x_1}& \ldots & \ldots & \ldots \\
        \ldots & \ldots & \ldots & \ldots \\
        f_{x_n x_1}& \ldots & \ldots & f_{x_n x_n} 
    \end{bmatrix}
\]

}
\defn{Derivate Pure}{
    Sono quelle che derivano per la stessa variabile (stanno sulla diagonale della matrice Hessiana):

    \[
        \frac{\partial^{2} f}{\partial x_i \partial x_i} = \frac{\partial^{2} f}{(\partial x_i)^{2}}
    \]

}
\defn{Derivate Miste}{ 

    \[
        \frac{\partial^{2} f}{\partial x_i \partial x_j}
    \]
}
\defn{Formula di Taylor con resto di Lagrange}{
   Sia $f \in \mathbb{C}^{k}(A)$, scriviamo la formula di Taylor di ordine $k-1$ con \textbf{resto di Lagrange}

   Osserviamo che $F(0) = f(x), F(1) = f(x+h)$. Esiste $\theta \in (0,1)$ tale che:

   \[
       F(1)  = F(0) + F'(0)(1-0) + \frac{F''(0)}{2}(1-0)^{2}+\ldots+ \frac{F^{k}(\theta)}{k!}(1-0)^{k} =
   \]

   \[
       = F(0) + F'(0) + \frac{F''(0)}{2}+\ldots+ \underbrace{\frac{F^{k}(\theta)}{k!}}_\text{resto di Lagrange}
   \]
}
\defn{}{

$f \in \mathbb{C}^{2}(A)$, stesse ipotesi di sopra. Allora:

\[
    f(x+h) = f(x) + \langle \nabla f(x),h \rangle + \frac{1}{2} \langle Hf(x) \cdot h, h \rangle + o(|h|)
\]

}
\defn{Resto di Peano in due variabili}{

    Dati $\bar{x} =(x_0,y_0)$, $\bar{h}  = (h,k)$:

    \[
        f(x_0 + h, y_0 + k) = f(x_0,y_0) + \frac{\partial f}{\partial x}(x_0,y_0)h + \frac{\partial f}{\partial x}(x_0,y_0) k +
    \]

    \[
        + \frac{1}{2} [ \frac{\partial^{2} f}{\partial x \partial x} (x_0,y_0)h^{2}+ 2 \frac{\partial^{2} f}{\partial x \partial y}(x_0,y_0)hk+ \frac{\partial^{2} f}{\partial y \partial y}(x_0,y_0) k ^{2}] + o(h^{2}+k^{2})
    \]

    Il pezzo tra parentesi quadre si verifica facendo i conti espliciti (prodotto scalare tra matrice per vettore e un vettore) considerando che $f_{xy} = f_{yx}$ perché $f \in \mathbb{C}^{2}$ (per il teorema di Schwarz).
    
}
\defn{Massimo e minimo locale}{

$f: D \subset \mathbb{R}^{n} \rightarrow  \mathbb{R}$, $D$ dominio (cioè aperto insieme alla sua frontiera). Si dice che $x_0 \in D$ è un punto di minimo locale se esiste un intorno sferico $B(x_0, \sigma) $ tale che:

\[
    f(x_0) \le f(x)
\]

$\forall x \in D \cap B(x_0, \sigma)$

La definizione è analoga per il punto di massimo locale.

}
\defn{Massimo/Minimo stretto}{

    Il punto di minimo o massimo locale si dice stretto se la disuguaglianza è stretta.
}
\defn{Massimo/Minimo globale}{ Se la disuguaglianza vale per tutto il dominio $\forall x \in D$ e non solo per la palla}
\defn{Punti stazionari}{
    I punti $x_0 \in A$ tali che $\nabla f(x_0) = 0$ si dicono \textbf{punti stazionari} (o \textbf{critici}). 
    }
\defn{Sella}{
L'essere punto stazionario è condizione necessaria ma non sufficiente per essere punto di estremo: in una variabile un punto stazionario che non era un punto di estremo si diceva flesso, in più variabili si parla di \textbf{sella}.
    }
\defn{}{ $q(\bar{h} )$ si dice \textbf{definita positiva} se $\forall h \neq 0$ si ha $q(\bar{h} ) >0$}
\defn{}{ $q(\bar{h} )$ si dice \textbf{definita negativa} se $\forall h \neq 0$ si ha $q(\bar{h} ) <0$}
\defn{}{ $q(\bar{h} )$ si dice \textbf{indefinita} se $\exists \bar{h_1}, \bar{h_2} \in \mathbb{R}^{2}$ t.c. $q(\bar{h_1} ) <0<q(\bar{h_2} )$ cioè cambia segno}
\defn{Somma inferiore}{ Definiamo somma inferiore di $f$ rispetto a $D$:

    \[
        s(f,D) = \sum^{n}_{i=1} \sum^{m}_{j=1} m_{ij} \cdot A_{ij}
    \]

    cioe' la somma dei parallelepipedi piccoli (vedi figura)

}
\defn{Somma superiore}{ Definiamo somma superiore di $f$ rispetto a $D$:

    \[
        S(f,D) = \sum^{n}_{i=1} \sum^{m}_{j=1} M_{ij} \cdot A_{ij}
    \]

    cioe' la somma dei parallelepipedi grandi (vedi figura)

}
\defn{Funzione integrabile secondo Riemann}{ Sia $f: R = [a,b] \times [c,d] \rightarrow \mathbb{R}$ limitata, si dice integrabile secondo Riemann se:

\[
    sup s(f,D) = inf S(f,D)
\]

In tal caso il valore comune si dice \textbf{integrale}  di $f$ su $R$ si indica in vari modi:

\[
    \int_{R}^{} {f}
\]

\[
    \iint_R {f}
\]

\[
    I(f,R)
\]

\[
    \iint_{R} f(x,y) dx dy 
\]

\[
    \int_{b}^{a} {\int_{d}^{c} {f(x,y) dx dy}} 
\]

}
\defn{}{$f$ è limitata su $D \subset \mathbb{R}^{2}$ con D insieme limitato di $\mathbb{R}^{2}$.

    Diciamo che $f$ è integrabile (secondo Reimann) su D, se la $\bar{f} $ è integrabile su $R$ (secondo Reimann) e in tal caso si scrive:

    \[
        \iint_D {f(x,y)} \: d x d y = \iint_R {\bar{f} (x,y)} \: dx d y 
    \]

}
\defn{Insieme numerabile (Peano-Jordan)}{Un sottoinsieme limitato del piano $D \subset \mathbb{R}^{2}$ si dice misurabile (secondo Peano-Jordan) se la funzione $f(x,y)=1$ è integrabile su D 

In tal caso poniamo:

\[
    Area(D) = \underbrace{|D|}_\text{area} = \iint {1} \: dx d y
\]

e dunque ogni rettangolo $R = [a,b] \times [c,d]$ è misurabile secondo Peano-Jordan:

\[
    |R| = \int_{a}^{b} {\int_{c}^{d} {1} \: dx } \: dx  = (b-a) (d-c)
\]
}
\defn{Vettore}{ Il vettore $n \in \mathbb{R}^{n}$ è una n-pla $x=(x_1,x_2, \ldots,x_n)$ }
\defn{Norma}{
        Il numero reale (non negativo)

        \[
          |x|:=  \sqrt{x \bullet x} = \sqrt{\langle x,x \rangle}
        \]


        si chiama \textbf{lunghezza} o \textbf{norma} del vettore

        }
\defn{Disuguaglianza triangolare}{
 La disuguaglianza triangolare si definisce come:

 \[
     |x+y| \le |x|+|y|
 \]
   
se $|x+y| = |x| + |y|$ $\rightarrow $ $y =0$ $\lor$ $x= \lambda y $ con $\lambda \ge 0$:

 }
\defn{Distanza Euclidea}{
Distanza Euclidea si definisce come $d(x,y)$:

\[
    d(x,y) := |x-y| = \sqrt{\langle x-y,x-y \rangle} = \sqrt{\sum^{n}_{i=1} (x_i - y_i)^{2}} \ge 0
\]

questa è la norma

}
\defn{Spazio metrico}{ Uno spazio metrico è un insieme X dotato di un'applicazione definita: $X \times X \rightarrow \mathbb{R}$ che verifica le proprietà sopra:

    \[
        (\mathbb{R}^{n},\underbrace{d}_\text{distanza euclidea}) \text{ spazio metrico}
    \]

    esistono altre distanze che ci definiscono relative metriche equivalenti

}
\defn{Successione}{ Una successione è un elenco ordinato di numeri $\{x_n\} \subset \mathbb{R}^{n}$ (gli elementi della successione sono elementi di $\mathbb{R}^{n}$ ovvero n-ple di reali)

    \[
        x_n= (x_n^{1}, x_n^{2},  \ldots , x_n ^{n})
    \]

    si dice che converge a $x \in \mathbb{R}^{n}$ se:

    \[
    d(x_n,x) \xrightarrow[] {n \rightarrow +\infty} 0
    \]

    cioè:

    \[
        \forall \varepsilon >0 , \exists \bar{N} \in \mathbb{N}
    \]

    \[
        d(x_n, x ) < \varepsilon, \forall n \ge \bar{N} 
    \]

    \[
        |x_n - x| = \sqrt{(x_n^{1}-x^{1})^{2}+  \ldots  +(x_n^{n}-x^{n})^{2}}
    \]

}
\defn{}{Si definisce palla aperta, disco aperto, intorno sferico di centro $x_0$ e raggio $r$ l'insieme e che si indica con $B(x_0,r)$ l'insieme:

    \[
        B(x_0,r) := \{ x \in \mathbb{R}^{n}, d(x,x_0)<r\} \subset \mathbb{R}^{n}
    \]

    praticamente un intorno di $x_0$ in $\mathbb{R}^{n}$


}
\defn{Sottoinsieme limitato}{$A \subset X$ si dice limitato se esiste una palla aperta in cui $A$ risulta interamente contenuto:

    \[
        \exists r>0,\exists x_0 \in X \text{ t.c. } A \subset B(x_0,r)
    \]
}
\defn{Punto interno}{

Un punto di $x_0 \in X$ si dice interno ad $A$ dove $A \subset X$ e non solo $x_0 \in A$ ma esiste (almeno) un suo intorno sferico interamente contenuto in $A$:

\[
    \exists r>0\ B(x_0,r) \subset A
\]

$\dot A$ insieme di punti interni ad $A$

}
\defn{Punto esterno}{ $x_0 \in X$ si dice esterno ad $A$ ($A \subset X$) se non solo $x_0$ non appartiene ad $A$ ma vi è almeno un suo intorno sferico completamente disgiunto ad $A$


    \[
        x_0 \in A \text{ e } \exists r>0\ B(x_0,r)\cap A = \emptyset
    \]

}
\defn{Insieme aperto}{$A \subset X$ si dice aperto se $A = \emptyset$ oppure se ogni punto è un punto interno di $A$ (ovvero per ogni punto di $A$ c'è un intorno sferico tutto contenuto in $A$)}
\defn{Insieme chiuso}{ Un insieme $C \subset X$ si chiuso se il suo complementare è un insieme aperto:

    \[
        X \setminus  C = A
    \]

}
\defn{Punto di accumulazione}{

    Un punto $x_0 \in R^{n}$ di accumulazione per $A \subseteq R^{n}$ si dice punto di accumulazione se in ogni intorno circolare di $x_0$ c'è almeno un punto di $A$ diverso da $x_0$

}
\defn{Convergenza in $\mathbb{R}^{n}$}{Data una successione $ \{x_n\}\in \mathbb{R}^{n}$ questa si dice che converge a $x_0 \in  \mathbb{R}^{n}$ se:

\[
    \lim_{ n \to +\infty } d(x_n,x_0)=0
\]

questo equivale a dire $\forall \varepsilon >0, \exists \bar{N} \in \mathbb{N}$ e $\forall n \ge \bar{N}$ si ha:

\[
    d(x_n,x_0) < \varepsilon
\]

}
\defn{Punto di accumulazione con limiti}{

    $x_0$ è di accumulazione per $A$ $\Leftrightarrow x_0$ è il limite di una successione di elementi di $A$ tutti diversi da $x_0$
}
\defn{Chiusura di un insieme $A \subset \mathbb{R}^{n}$}{

    Si indica con $\bar{A} $ è un sottoinsieme di $\mathbb{R}^{n}$ dato dall'unione di $A$ e dei suoi punti di accumulazione ($DA$)

    $\bar{A} $ è un insieme chiuso. Lo si può pensare come l'intersezione dei chiusi contenenti $A$.

    Si può inoltre dimostrare che:

    \[
        \bar{A} = A \cup \delta A
    \]

}
\defn{Dominio}{ Un dominio $D$ in $\mathbb{R}^{n}$ è la chiusura di un insieme aperto:

    \[
        D = \bar{A}  = A \cup \delta A
    \]
}
\defn{Limite di funzioni in più variabili}{

Sia $x_0 \in \mathbb{R}^{n}$ un punto di accumulazione per $A$

Si dice che $f(\bar{x})$ tende (ha limite) a $l$ per $\bar{x} $ che tende a $x_0$:

\[
    \lim_{ \bar{x}  \to x_0 } f(\bar{x} ) = l
\]

scrivendolo tramite gli intorni: se $\forall $ intorno $U \subset \mathbb{R}$ di $l$ esiste un intorno di $x_0$ (sferico) $I(x_0,r)$ con $r>0$ 

tale che $f(\bar{x}) \in U$ $\forall \bar{x}   \in \underbrace{I(x_0,r)}_{B(x_0,r)}\cap (A \setminus \{x_0\})$ 

L'altra definizione con i delta:

\[
    \forall \varepsilon>0\  \exists \delta >0
\]

tale che 

\[
    \underbrace{|f(\bar{x} ) - l|}_{d(f( \bar{x} ),l) \in \mathbb{R}} <\varepsilon
\]

$\forall \bar{x} \in A \setminus \{x_0\}$ con $|\bar{x} -x_0| < \delta$
}
\end{document}
