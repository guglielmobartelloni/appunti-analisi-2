\documentclass[11pt]{article}


\documentclass[11pt]{article}
\usepackage[italian]{babel}
\usepackage[utf8]{inputenc}	% Para caracteres en español
\usepackage{amsmath,amsthm,amsfonts,amssymb,amscd}
\usepackage{multirow,booktabs}
\usepackage[table]{xcolor}
\usepackage{fullpage}
\usepackage{lastpage}
\usepackage{enumitem}
\usepackage{fancyhdr}
\usepackage{mathrsfs}
\usepackage{wrapfig}
\usepackage{graphicx}
\graphicspath{ {./images/} }
\usepackage{setspace}
\usepackage{calc}
\usepackage{multicol}
\usepackage{cancel}
\usepackage[retainorgcmds]{IEEEtrantools}
\usepackage[margin=3cm]{geometry}
\usepackage{amsmath}
\newlength{\tabcont}
\setlength{\parindent}{0.0in}
\setlength{\parskip}{0.05in}
\usepackage{empheq}
\usepackage{framed}
\usepackage[most]{tcolorbox}
\usepackage{xcolor}
\usepackage{FiraSans}
\usepackage{listings}
\usepackage{color} %red, green, blue, yellow, cyan, magenta, black, white
\definecolor{mygreen}{RGB}{28,172,0} % color values Red, Green, Blue
\definecolor{mylilas}{RGB}{170,55,241}
\colorlet{shadecolor}{orange!15}
\parindent 0in
\parskip 12pt
\geometry{margin=1in, headsep=0.25in}
\theoremstyle{definition}
\newtheorem{defn}{Definizione}
\newtheorem{reg}{Regola}
\newtheorem{oss}{Osservazione}
\newtheorem{exer}{Esecizio}
\newtheorem{note}{Nota}
\newtheorem{thm}{Teorema}[section] % reset theorem numbering for each chapter
\theoremstyle{plain}
\newcommand{\restr}[2]{%
\mathchoice{%
\restriction{#1}{#2}{\displaystyle}%
}{%
\restriction{#1}{#2}{\textstyle}%
}{%
\restriction{#1}{#2}{\scriptstyle}%
}{%
\restriction{#1}{#2}{\scriptscriptstyle}%
}%
}

\lstset{language=Matlab,%
    %basicstyle=\color{red},
    breaklines=true,%
    morekeywords={matlab2tikz},
    keywordstyle=\color{blue},%
    morekeywords=[2]{1}, keywordstyle=[2]{\color{black}},
    identifierstyle=\color{black},%
    stringstyle=\color{mylilas},
    commentstyle=\color{mygreen},%
    showstringspaces=false,%without this there will be a symbol in the places where there is a space
    numbers=left,%
    numberstyle={\tiny \color{black}},% size of the numbers
    numbersep=9pt, % this defines how far the numbers are from the text
    emph=[1]{for,end,break},emphstyle=[1]\color{red}, %some words to emphasise
    %emph=[2]{word1,word2}, emphstyle=[2]{style},    
}


\newlength{\totbarheight}
\newlength{\bardepth}



\begin{document}

\title{Lezione 2}
\author{Guglielmo Bartelloni}

\maketitle
\tableofcontents
\newpage
\thispagestyle{empty}

\section{Facciamo vedere che il teorema precedente valeva anche per $n>1$}

Supponiamo che $u$ e $v$ siamo due soluzioni di (1), cioè che:

$Lu=f$ e $Lv=f$ su $I$

La differenza di queste diventano soluzione su $I=[a,b]$ dell'omogenea associata

Usando la propietà della linearità:

\[
    L(\lambda u+\mu v) = \lambda L u + \mu L v
    
\]

\[
    L(u-v) = Lu-Lv = f- f=0
\]

Se indichiamo con $V_0$ l'insieme di tutte le soluzioni dell'equazione omogenea associata ($Lw=0$ su $I=[a,b]$ e $V_0$ è l'insieme delle $w \in \mathbb{C}^n(I)$) e con $\bar u(t)$ una soluzione nota di (1)

\[
    u(x) = \bar u(x) +w(x)
\]

L'uguaglianza sopra, al variare di $w(x)$ in $V_0$ ci da tutte le soluzioni del problema di partenza. 

(Il problema quindi, diventa solo di studiare il problema omogeneo)

\section{Torniamo al I ordine}

Adesso ritorniamo al problema di I ordine (in forma normale):

\[
    (1)\ y'(x)+a(x)y(x)=f(x)
\]

dove $a()$ e $f()$ sono continue su $[a,b]$

\[
    (2)\ y'(x)+a(x)y(x)=0
\]

Secondo il teorema della prima lezione:

\[
    y(x)=z(x)+\bar y(x)
\]

Come si determina l'insieme di tutte le soluzioni (integrale generale) di (2), cioè:

\[
    (2)\ y'(x)+a(x)y(x)=0,x \in [a,b]
\]

Sia $A(x)$ una \textbf{primitiva} di $a(x)$:

\[
    A(x) = \int_{{}}^{{}} {a(x)} \: d{x} {}
\]

Moltiplichiamo i due membri della (2) per $e^{A(x)}$:

\[
    e^{A(x)} y'(x) + e ^{A(x)}a(x) y(x)=0, x \in [a,b]
\]

La posso scrivere anche (la derivata di $e ^{A(x)}y(x)$):

\[
    (e ^{A(x)} y(x))' = e ^{A(x)}a(x)y(x) + e ^{A(x)}y'(x)
\]

quindi (sempre chiaramente nell'intervallo $[a,b]$):

\[
    (e ^{A(x)}y(x))'=0
\]

Questo mi dice che:

\[
    e ^{A(x)}y(x) = costante=c \in \mathbb{R}
\]

porto dall'altra parte:

\[
    y(x) = c e ^{-A(x)}
\]

espandendo $A(x)$:

\[
    y(x) = c e ^{\int_{{}}^{{}} {a(x)} \: d{x} {}}
\]

posso considerare le soluzioni come:

\[
    y(x) = c z_0
\]

dove $z_0$ è una soluzione particolare di (2).

Infatti $e ^{-A(x)}$ è soluzione di (2)

\begin{proof}
    \[
    e ^{-A(x)} = -a(x) e ^{-A(x)}
\]

ovvero

\[
    (e ^{-A(x)})'+a(x) e ^{-A(x)}=0
\]
    
\end{proof}


\subsection{Determinazione dell'integrale particolare}

Sappiamo:

\[
    (1)\ y'(x)+a(x)y(x)=f(x)
\]

\[
    (2)\ y'(x)+a(x)y(x)=0
\]

Cerco l'integrale particolare ad \textbf{occhio} oppure uso il \textbf{metodo della variazione della costante}

\subsubsection{Metodo della variazione della costante}

Cerco questa $c(x)$ in questa forma:

\[
    \bar y(x) = c(x) e ^{-A(x)}
\]

Ovviamente la cerco dopo che so che $\bar y(x)$ è soluzione del problema.

\begin{proof}
    Poichè $\bar y(x)$ è soluzione di (1) si ha che $\bar y'(x)+a(x) \bar y(x)=f(x)$ da cui sostituendo $\bar y(x) = c(x) e ^{-A(x)}$:

    \[
        (c(x) e ^{-A(x)})'+ a(x) c(x) e ^{-A(x)} = f(x)
    \]
    
    Deriviamo:

    \[
        c'(x) e ^{-A(x)} - c(x) a(x) e ^{-A(x)} + a(x) c(x) e ^{-A(x)}= f(x)
    \]
    
    semplifico

    \[
        c'(x) e ^{-A(x)} = f(x)
    \]

    \[
        c'(x) = f(x) e ^{A(x)} \rightarrow c(x) = \int_{{}}^{{}} {f(x) e ^{A(x)}} \: d{} {}
    \]

    e dunque:

    \[
        \bar y(x) = e ^{-A(x)} \int_{{}}^{{}} {f(x) e ^{A(x)}} \: d{x} {}
    \]

    Cioè l'integrale particolare
\end{proof}

Se metto tutto insieme l'integrale generale diventa:

\[
    y(x) = c e ^{-A(x)} + e ^{-A(x)} \int_{{}}^{{}} {f(x) e ^{A(x)}} \: d{x} {}
\]

\subsection{Osservazioni sulla formula}

$A(x)$ è \textbf{una} primitiva di $a(x)$ scelta una volta per tutte.

\textbf{Non} occorre mettere una costante arbitraria (ovvero considerare come $A(x) + K,K \in \mathbb{R}$ ) poiche l'integrale generale non cambia

\textbf{Non} serve neanche nell'integrale perchè verrebbe buttato dentro $c$ dell'integrale generale


\subsection{Esempi}

\[
    y'(x) = 5y(x) + e ^{x}
\]

in questo caso $a(x) = -5$

\[
    A(x) = - \int_{{}}^{{}} {5} \: d{x} {}=-5x
\]

Quindi: 

\[
    e ^{-A(x)}=e ^{5x}
\]

\[
    y(x) = c e ^{5x} + e ^{5x} \int_{{}}^{{}} {e^x e ^{-5x}} \: d{x} {} = c e ^{5x} + e ^{5x} \int_{{}}^{{}} {e ^{-4x}} \: d{} {} = c e ^{5x} + e ^{5x} (\frac{1}{4} e ^{-4x}) = c e ^{5x} - \frac{1}{4} e ^{x}
\]

Esercizio per casa:

\[
    u' + \frac{u}{t} = e ^{t}
\]

\end{document}
