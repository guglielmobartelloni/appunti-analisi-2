\documentclass[11pt]{article}


\documentclass[11pt]{article}
\usepackage[italian]{babel}
\usepackage[utf8]{inputenc}	% Para caracteres en español
\usepackage{amsmath,amsthm,amsfonts,amssymb,amscd}
\usepackage{multirow,booktabs}
\usepackage[table]{xcolor}
\usepackage{fullpage}
\usepackage{lastpage}
\usepackage{enumitem}
\usepackage{fancyhdr}
\usepackage{mathrsfs}
\usepackage{wrapfig}
\usepackage{graphicx}
\graphicspath{ {./images/} }
\usepackage{setspace}
\usepackage{calc}
\usepackage{multicol}
\usepackage{cancel}
\usepackage[retainorgcmds]{IEEEtrantools}
\usepackage[margin=3cm]{geometry}
\usepackage{amsmath}
\newlength{\tabcont}
\setlength{\parindent}{0.0in}
\setlength{\parskip}{0.05in}
\usepackage{empheq}
\usepackage{framed}
\usepackage[most]{tcolorbox}
\usepackage{xcolor}
\usepackage{FiraSans}
\usepackage{listings}
\usepackage{color} %red, green, blue, yellow, cyan, magenta, black, white
\definecolor{mygreen}{RGB}{28,172,0} % color values Red, Green, Blue
\definecolor{mylilas}{RGB}{170,55,241}
\colorlet{shadecolor}{orange!15}
\parindent 0in
\parskip 12pt
\geometry{margin=1in, headsep=0.25in}
\theoremstyle{definition}
\newtheorem{defn}{Definizione}
\newtheorem{reg}{Regola}
\newtheorem{oss}{Osservazione}
\newtheorem{exer}{Esecizio}
\newtheorem{note}{Nota}
\newtheorem{thm}{Teorema}[section] % reset theorem numbering for each chapter
\theoremstyle{plain}
\newcommand{\restr}[2]{%
\mathchoice{%
\restriction{#1}{#2}{\displaystyle}%
}{%
\restriction{#1}{#2}{\textstyle}%
}{%
\restriction{#1}{#2}{\scriptstyle}%
}{%
\restriction{#1}{#2}{\scriptscriptstyle}%
}%
}

\lstset{language=Matlab,%
    %basicstyle=\color{red},
    breaklines=true,%
    morekeywords={matlab2tikz},
    keywordstyle=\color{blue},%
    morekeywords=[2]{1}, keywordstyle=[2]{\color{black}},
    identifierstyle=\color{black},%
    stringstyle=\color{mylilas},
    commentstyle=\color{mygreen},%
    showstringspaces=false,%without this there will be a symbol in the places where there is a space
    numbers=left,%
    numberstyle={\tiny \color{black}},% size of the numbers
    numbersep=9pt, % this defines how far the numbers are from the text
    emph=[1]{for,end,break},emphstyle=[1]\color{red}, %some words to emphasise
    %emph=[2]{word1,word2}, emphstyle=[2]{style},    
}


\newlength{\totbarheight}
\newlength{\bardepth}



\begin{document}

\title{Introduzione}
\author{Guglielmo Bartelloni}

\maketitle
\tableofcontents
\newpage
\thispagestyle{empty}

\section{Equazioni differenziali}

Le equazioni differenziali sono equazioni in cui l'ingnita è un equazione insieme a qualche sua derivata.

\subsection{Equazioni differenziali ordinarie}

Noi vedremo quelle del primo ordine lineari e di secondo ordine con coefficienti costanti

Problema di Cauchy: problema con codizioni iniziali.


\defn{}{Una equazione di ordine n è una equazione del tipo:
\[
    F(x,y(x),y'(x),...,y^{(n-1)}(x),y^{(n)}(x))=0
\]
\[
    u \in I contenuto \mathbb{R}
\]
dove l'incognita è la qualunque y(x). F è funzione di (n+2) variabili $x,y(x),y'(x)....$
}



L'\textbf{ordine} è dato dal massimo ordine di derivazione che compare.


Per esempio:
\[
    y'''+2y''+5y = e^x
\]
è di ordine 3

\defn{Soluzione (curva) integrale}{La soluzione di una EDO di ordine n sull'intervallo I \[
        (*) F(x,y(x),y'(x),...) = 0
\]
\[
    x \in I contenuto \mathbb{R}
\]

$\phi(x)$ che sia definita (almeno) in I e ivi derivabile fino all'ordine n per cui valga (*), ovvero:
\[
    F(x,\phi(x),\phi ' (x), ... ) = 0 
\]

$\forall x \in I$

Chiaramente cambia a seconda dell'intervallo
}

\defn{Integrale Generale}{Si chiama integrale \textbf{generale} di (*) in I l'insieme di tutte le soluzioni di (*) in I}


E' possibile definire un esepressione piu' esplicita

\defn{Forma normale}{Un edo di ordine n si dice in forma normale se è in forma

    \[
        y^{(n)} = f(x,y(x),y'(x),....,y^{(n-1)}, x \in I
    \]
    
    Esempio:
    \[
        y'''=-5y'+sinx
    \]
    Quella sopra è un EDO di III ordine normale.
}

\defn{EDO di ordine n lineare}{Una EDO di ordine n si dice lineare se è nella forma
    \[
        a_n(x)y^{(n)}(x)+a_{n-1}(x)y^{(n-1)}+... + a_2(x)y''(x)+a_1(x)y'(x)+a_0y(x)=f(x),x \in I
    \]

    Dove le funzioni \[
        a_0(x),a_1(x),a_2(x),...,a_n(x),f(x)
    \]

    sono assegnate (continue) in I

    Esempio:
    \[
        xy''+5y = sin x
    \]

}


Quando $f(x)=0$ allora l'equazione si dice l'\textbf{omogenea associata} 


Nel nostro caso le equazioni di secondo ordine lineari saranno a \textbf{coefficienti costanti}






\end{document}
