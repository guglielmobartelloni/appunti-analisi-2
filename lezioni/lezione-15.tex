\documentclass[../appunti-analisi.tex]{subfiles}

\begin{document}

\section{Lezione 15}

\subsection{Differenziabilità}

Per dire che $z = f(x_0,y_0) + f_x(x_0,y_0) (x-x_0) + f_y(x_0,y_0) (y-y_0) $ è l'equazione del piano tangente, devo vedere anche la differenziabilità (il piano contiene le rette $T_1$ e $T_2$ che hanno in comune $P_0$).


Il fatto che la funzione ad \textbf{una variabile} sia derivabile in un punto mi dice che esiste la retta tangente in quel punto.


Nelle due variabili questo non accade, bisogna vedere altro.

Prima parliamo delle curve:

\subsubsection{Curve in $\mathbb{R}^{n}$}

\defn{Curva}{
Una curva è un'applicazione continua:

\[
    \varphi: I \subset \mathbb{R} \rightarrow \mathbb{R}^{n}
\]

per $I=[a,b]$:

\[
   \bar{\varphi}(t) = (\varphi_1(t),\varphi_2(t), \ldots ,\varphi_n(t)) \in \mathbb{R}^{n}
\]

Le equazioni parametriche sono:

$\varphi=$\begin{equation}
    \begin{cases}
           x_1(t)=\varphi_1(t)\\
           x_2(t) = \varphi_2(t)\\
   &\;\;\vdots \notag \\
           x_n(t) = \varphi_n(t)

    \end{cases}
\end{equation}

}


l'immagine della curva è chiamata sostegno della curva:

\[
    \varphi(I) \subset \mathbb{R}^{n}
\]

\textbf{Esempi} 

Curve cartesiane

Sia $f:[a,b]\rightarrow \mathbb{R}$ continua il suo grafico è il sostegno della curva piana data da:

\[
    \varphi:[a,b]\rightarrow \mathbb{R}^{2}
\]

\[
    \varphi(t) = (\varphi_1(t), \varphi_2(t))
\]

definita da:

\begin{equation}
    \begin{cases}
           x_1(t) = t\\
           x_2(t) = f(t) 
    \end{cases}\,.
\end{equation}


\defn{Curva regolare}{
Una curva si dice regolare se l'applicazione $\varphi$ è di classe $C^{1}$ (le derivate prime sono continue) e $\varphi'(t) \neq 0$

In particolare $\varphi'(t) \neq 0$ significa che il vettore:

\[
    (\varphi_1'(t), \ldots ,\varphi_n'(t)) = \varphi'(t)
\]

non ha mai tutte le componenti contemporaneamente nulle.
}


\textbf{Esempio} 

Se $f \in \mathbb{C}^{1}([a,b])$ e $\varphi'(t) = (1,f'(t))$

allora la retta tangente alla curva nel punto $\varphi(t_0)$ è proprio la retta tangente al grafico di $f$ nel punto $(x_0,f(x_0))=(t_0,f(t))$

Se $\varphi(t)$ è regolare allora il vettore $\varphi'(t_0)$ si chiama vettore tangente alla curva nel punto $\varphi(t_0) \in \mathbb{R}^{3}$

\textbf{Esempio di due curve con stesso sostegno} 

Consideriamo le due applicazioni a valori vettoriali:

\[
    \underbrace{\varphi:[0,2\pi] \rightarrow \mathbb{R}^{2}}_{\varphi(t) = (\varphi_1(t),\varphi_2(t))}
\]

\[
    \underbrace{\psi: [0,4\pi] \rightarrow  \mathbb{R}^{2}}_{\psi(t) = (\psi_1(t),\psi_2(t))}
\]


\begin{equation}
    \begin{cases}
           \varphi_1(t) = \cos t\\
           \varphi_2(t) = \sin t
    \end{cases}\,.
\end{equation}


\begin{equation}
    \begin{cases}
           \psi_1(t) = \cos t\\
           \psi_2(t) = \sin t
    \end{cases}\,.
\end{equation}

Il sostegno delle due curve è lo stesso vanno entrambe a $\mathbb{R}^{2}$

Le curve però sono diverse perché il dominio è diverso.


Possiamo scrivere le derivati parziali con notazione vettoriale.

\[
    f: A\subseteq \mathbb{R}^{n}\rightarrow \mathbb{R} \text{ con A aperto}
\]

\textbf{Scrittura alternativa della derivata parziale} 

Quindi posso scrivere la derivata parziale:

\[
f_{x_i} (\bar{x} ) = \frac{\partial f}{\partial x_i}(\bar{x} )
\]

nel modo seguente, definisco $\forall i = 1, \ldots ,n$:

\[
    \bar{e_i}  = (0, \ldots 0,1,0,....,0) 
\]

quindi la derivata è data dal seguente limite, purché esista e sia finito:

\[
\lim_{ h \to 0 } \frac{f(\bar{x} + h \bar{e_i}) - f(\bar{x} )}{h} = f_{x_i}(\bar{x} )
\]

espandendo:

\[
    f( \bar{x} +h \bar{e_i} ) = f( x_1,...,x_{i-1},x_i+h , x_{i+1},...,x_n)
\]

vediamo che dipende solo da h:

\[
    g(h) = f(\bar{x} ,h \bar{e_i} )
\]

Se g è derivabile allora si ha:

\[
    g'(0) = \frac{\partial f}{\partial x_i}(\bar{x} )
\]

perché espandendo:

\[
    g'(0) = \lim_{ h \to 0 } \frac{g(h) - g(0)}{h-0} = \lim_{ h \to 0 } \frac{g(\bar{x} +h \bar{e_i} ) - f(\bar{x} )}{h}
\]

se io ho $\bar{v} $ direzione in $\mathbb{R}^{n}$ di modulo 1:

\[
    |\bar{v} | = \sqrt{\sum^{n}_{i=1} v_i^{2}}
\]

Derivata direzionale:

\[
    D_v f(\bar{x} ) = \lim_{ h \to 0 } \frac{f(\bar{x} +h \bar{v} ) -f(\bar{x} )}{h}
\]


\textbf{Esempio} 

\[
    f: A \subseteq \mathbb{R}^{3} \rightarrow  \mathbb{R}
\]

\[
    \bar{v} = (a,b,c) \in \mathbb{R}^{3}
\]

\[
    a^{2}+b^{2}+c^{2}= 1\ (||\bar{v} || =1)
\]

Calcoliamo la derivata direzionale:

\[
    D_v f(\bar{x_0} ) = \lim_{ h \to 0 } \frac{f(\bar{x_0} + h \bar{v} ) -f(\bar{x_0} )}{h}
\]

dove:

\[
    \bar{x_0} = (x_0,y_0,z_0)
\]

\[
    D_v f(P_0) = D_v f(\bar{x_0} ) = \lim_{ h \to 0 } \frac{f(x_0+ha,y_0+hb,z_0+hc) - f(x_0,y_0,z_0)}{h}
\]

questo esiste se il limite esiste ed è finito.

dire che quindi $f$ è derivabile in $\bar{x} \in A$ è come dire che esiste il vettore gradiente:

\[
   \nabla f(\bar{x} ) 
\]

\end{document}
