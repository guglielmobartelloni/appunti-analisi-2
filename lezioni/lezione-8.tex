\documentclass[../appunti-analisi.tex]{subfiles}

\begin{document}

\section{Lezione 8}

La distanza euclidea verifica le proprietà (le stesse del valore assoluto):

\begin{enumerate}
    \item  $d(x,y) \ge 0, d(x,y) \in \mathbb{R}$
    \item  $d(x,y) = 0 \Leftrightarrow x=y$
    \item  $d(x,y) = d(y,x)$ simmetria
    \item  $d(x,y) \le d(x,z) + d(z,y), \forall x,y,z \in \mathbb{R}^{n} $ disuguaglianza triangolare
\end{enumerate}

\defn{Spazio metrico}{ Uno spazio metrico è un insieme X dotato di un'applicazione definita: $X \times X \rightarrow \mathbb{R}$ che verifica le proprietà sopra:

    \[
        (\mathbb{R}^{n},\underbrace{d}_\text{distanza euclidea}) \text{ spazio metrico}
    \]

    esistono altre distanze che ci definiscono relative metriche equivalenti

}


\textbf{Esempio} 

\[
    \mathbb{R}^{2},\forall x,y \in \mathbb{R}^{2}
\]

\[
    x=(x_1,x_2),y=(y_1,y_2)
\]

\[
    d(x,y) = \sqrt{(x_1-y_1)^{2}+(x_2-y_2)^{2}}
\]

la distanza si può scrivere anche:
   
\[
    d_1(x,y) = |x_2-x_1| + |y_2-y_1| \text{ dove abbiamo usato i valori assoluti}
\]

\newpage

\subsection{Successioni convergenti in $\mathbb{R}^{n}$}

\defn{Successione}{ Una successione è un elenco ordinato di numeri $\{x_n\} \subset \mathbb{R}^{n}$ (gli elementi della successione sono elementi di $\mathbb{R}^{n}$ ovvero n-ple di reali)

    \[
        x_n= (x_n^{1}, x_n^{2},  \ldots , x_n ^{n})
    \]

    si dice che converge a $x \in \mathbb{R}^{n}$ se:

    \[
    d(x_n,x) \xrightarrow[] {n \rightarrow +\infty} 0
    \]

    cioè:

    \[
        \forall \varepsilon >0 , \exists \bar{N} \in \mathbb{N}
    \]

    \[
        d(x_n, x ) < \varepsilon, \forall n \ge \bar{N} 
    \]

    \[
        |x_n - x| = \sqrt{(x_n^{1}-x^{1})^{2}+  \ldots  +(x_n^{n}-x^{n})^{2}}
    \]

}

\proposizione{}{ Sia $\{x_n\} \subset \mathbb{R}^{n}$ una successione in $\mathbb{R}^{n}$. Si ha:

    \begin{enumerate}
        \item (\textbf{Unicità}) $\{x_n\}$ ha massimo un unico limite (se $\{x_n\}$ ammette limite questo è unico)
        \item (\textbf{Limitatezza}) ogni successione convergente è limitata: 

             \[
                 \exists x_0 \in \mathbb{R}^{n},M \in \mathbb{R}
             \]

             \[
                 d(x_n,x_0) \le M, \forall n
             \]

         \item (\textbf{Sottosuccessione}) Se $\{x_n\}$ convergente a $x \in  \mathbb{R}^{n}$ (o in $X$) allora ogni sottosuccessione ${x_{n_k}}$ estratta da ${x_n}$ converge allo stesso limite
    \end{enumerate}

}

\subsection{Elementi di topologia in $\mathbb{R}^{n}$ (in X)}

Sia $x_0 \in \mathbb{R}^{n}$ fissato e $r>0$

Si ha la seguente definizione

\defn{}{Si definisce palla aperta, disco aperto, intorno sferico di centro $x_0$ e raggio $r$ l'insieme e che si indica con $B(x_0,r)$ l'insieme:

    \[
        B(x_0,r) := \{ x \in \mathbb{R}^{n}, d(x,x_0)<r\} \subset \mathbb{R}^{n}
    \]

    praticamente un intorno di $x_0$ in $\mathbb{R}^{n}$


}

\begin{figure}[ht]
    \centering
    \incfig{disco-aperto}
    \caption{disco aperto}
    \label{fig:disco-aperto}
\end{figure}

\textbf{Esempio in $(\mathbb{R}^{2},d)$ e $(\mathbb{R}^{2},d_1)$} 

\[
    d(x,y) = \sqrt{\sum^{r}_{i=1} (x_i-y_i)^{2}}
\]

\[
    d_1(x,y) = |x_1-y_1| + |x_2-y_2|
\]

se $x_0=0$ e $r=1$

\[
    B(0,1)  = \{\bar{x} \in \mathbb{R}^{2}, d(x,0) <1\} = \{x \in  \mathbb{R}^{2}, \sqrt{x_1^{2}+y_1^{2}}<1\}
\]

questo è un cerchio


\[
    B(0,1)  = \{\bar{x} \in \mathbb{R}^{2}, d_1(x,0) <1\} = \{x \in  \mathbb{R}^{2}, \underbrace{|x_1-0|}_{x_1} + \underbrace{|y_1-0|}_{y_1}<1\}
\]

questo è un rombo (parallelogramma).


Abbiamo parlato di questa roba per definire i limiti attraverso gli intorni sferici

Quindi:

\[
\{ x_n\} \subset X \text{ converge a } x_0 \in X \Leftrightarrow \forall \varepsilon >0 , \exists n_\varepsilon, x_n \in B(x_0,\varepsilon), \forall n \ge n_\varepsilon
\]

\defn{Sottoinsieme limitato}{$A \subset X$ si dice limitato se esiste una palla aperta in cui $A$ risulta interamente contenuto:

    \[
        \exists r>0,\exists x_0 \in X \text{ t.c. } A \subset B(x_0,r)
    \]
}

\defn{Punto interno}{

Un punto di $x_0 \in X$ si dice interno ad $A$ dove $A \subset X$ e non solo $x_0 \in A$ ma esiste (almeno) un suo intorno sferico interamente contenuto in $A$:

\[
    \exists r>0\ B(x_0,r) \subset A
\]

$\dot A$ insieme di punti interni ad $A$

}

\defn{Punto esterno}{ $x_0 \in X$ si dice esterno ad $A$ ($A \subset X$) se non solo $x_0$ non appartiene ad $A$ ma vi è almeno un suo intorno sferico completamente disgiunto ad $A$


    \[
        x_0 \in A \text{ e } \exists r>0\ B(x_0,r)\cap A = \emptyset
    \]

}


I punti che non sono ne esterni ne interni si dicono di frontiera.

\newpage

\defn{Insieme aperto}{$A \subset X$ si dice aperto se $A = \emptyset$ oppure se ogni punto è un punto interno di $A$ (ovvero per ogni punto di $A$ c'è un intorno sferico tutto contenuto in $A$)}


\defn{Insieme chiuso}{ Un insieme $C \subset X$ si chiuso se il suo complementare è un insieme aperto:

    \[
        X \setminus  C = A
    \]

}


\textbf{Esempio} 

Sia

\[
    A_2= \{ (x,y) \in \mathbb{R}^{2}, x \neq y^{2}\} \subset \mathbb{R}^{2}
\]

\begin{figure}[ht]
    \centering
    \incfig{esercizio-su-insieme-chiuso-1}
    \caption{esercizio su insieme chiuso 1}
    \label{fig:esercizio-su-insieme-chiuso-1}
\end{figure}


\textbf{Esercizio per casa} 

\[
    A_1= \{ (x,y) \in \mathbb{R}^{2}, x^{2}+ y^{2} < 25\}
\]

\[
    A_2= \{ (x,y) \in \mathbb{R}^{2}, x^{2}+ 2y +1  \le  0\}
\]

\end{document}
