\documentclass[11pt]{article}


\documentclass[11pt]{article}
\usepackage[italian]{babel}
\usepackage[utf8]{inputenc}	% Para caracteres en español
\usepackage{amsmath,amsthm,amsfonts,amssymb,amscd}
\usepackage{multirow,booktabs}
\usepackage[table]{xcolor}
\usepackage{fullpage}
\usepackage{lastpage}
\usepackage{enumitem}
\usepackage{fancyhdr}
\usepackage{mathrsfs}
\usepackage{wrapfig}
\usepackage{graphicx}
\graphicspath{ {./images/} }
\usepackage{setspace}
\usepackage{calc}
\usepackage{multicol}
\usepackage{cancel}
\usepackage[retainorgcmds]{IEEEtrantools}
\usepackage[margin=3cm]{geometry}
\usepackage{amsmath}
\newlength{\tabcont}
\setlength{\parindent}{0.0in}
\setlength{\parskip}{0.05in}
\usepackage{empheq}
\usepackage{framed}
\usepackage[most]{tcolorbox}
\usepackage{xcolor}
\usepackage{FiraSans}
\usepackage{listings}
\usepackage{color} %red, green, blue, yellow, cyan, magenta, black, white
\definecolor{mygreen}{RGB}{28,172,0} % color values Red, Green, Blue
\definecolor{mylilas}{RGB}{170,55,241}
\colorlet{shadecolor}{orange!15}
\parindent 0in
\parskip 12pt
\geometry{margin=1in, headsep=0.25in}
\theoremstyle{definition}
\newtheorem{defn}{Definizione}
\newtheorem{reg}{Regola}
\newtheorem{oss}{Osservazione}
\newtheorem{exer}{Esecizio}
\newtheorem{note}{Nota}
\newtheorem{thm}{Teorema}[section] % reset theorem numbering for each chapter
\theoremstyle{plain}
\newcommand{\restr}[2]{%
\mathchoice{%
\restriction{#1}{#2}{\displaystyle}%
}{%
\restriction{#1}{#2}{\textstyle}%
}{%
\restriction{#1}{#2}{\scriptstyle}%
}{%
\restriction{#1}{#2}{\scriptscriptstyle}%
}%
}

\lstset{language=Matlab,%
    %basicstyle=\color{red},
    breaklines=true,%
    morekeywords={matlab2tikz},
    keywordstyle=\color{blue},%
    morekeywords=[2]{1}, keywordstyle=[2]{\color{black}},
    identifierstyle=\color{black},%
    stringstyle=\color{mylilas},
    commentstyle=\color{mygreen},%
    showstringspaces=false,%without this there will be a symbol in the places where there is a space
    numbers=left,%
    numberstyle={\tiny \color{black}},% size of the numbers
    numbersep=9pt, % this defines how far the numbers are from the text
    emph=[1]{for,end,break},emphstyle=[1]\color{red}, %some words to emphasise
    %emph=[2]{word1,word2}, emphstyle=[2]{style},    
}


\newlength{\totbarheight}
\newlength{\bardepth}



\begin{document}

\title{Lezione 2}
\author{Guglielmo Bartelloni}

\maketitle
\tableofcontents
\newpage
\thispagestyle{empty}

\section{Facciamo vedere che il teorema precedente valeva anche per $n>1$}

Supponiamo che $u$ e $v$ siamo due soluzioni di (1), cioè che:

$Lu=f$ e $Lv=f$ su $I$

La differenza di queste diventano soluzione su $I=[a,b]$ dell'omogenea associata

Usando la propietà della linearità:

\[
    L(\lambda u+\mu v) = \lambda L u + \mu L v
    
\]

\[
    L(u-v) = Lu-Lv = f- f=0
\]

Se indichiamo con $V_0$ l'insieme di tutte le soluzioni dell'equzione omogenea associata ($Lw=0$ su $I=[a,b]$ e $V_0$ è l'insieme delle $w \in \mathbb{C}^n(I)$) e con $\bar u(t)$ una soluzione nota di (1)

\[
    u(x) = \bar u(x) +w(x)
\]

L'uguglianza sopra, al variare di $w(x)$ in $V_0$ ci da tutte le soluzioni del problema di partenza. 

(Il problema quindi, diventa solo di studiare il problema omogeneo)

\section{Torniamo al I ordine}

Adesso ritorniamo al problema di I ordine (in forma normale):

\[
    (1)\ y'(x)+a(x)y(x)=f(x)
\]

dove $a()$ e $f()$ sono continue su $[a,b]$

\[
    (2)\ y'(x)+a(x)y(x)=0
\]

Secondo il teorema della prima lezione:

\[
    y(x)=z(x)+\bar y(x)
\]

Come si determina l'insieme di tutte le soluzioni (integrale generale) di (2), cioe:

\[
    (2)\ y'(x)+a(x)y(x)=0,x \in [a,b]
\]

Sia $A(x)$ una \textbf{primitiva} di $a(x)$:

\[
    A(x) = \int_{{}}^{{}} {a(x)} \: d{x} {}
\]

Moltiplichiamo i due membri della (2) per $e^{A(x)}$:

\[
    e^{A(x)} + e ^{A(x)}a(x) y(x)=0, x \in [a,b]
\]

La posso scrivere anche (la derivata di $e ^{A(x)}y(x)$):

\[
    (e ^{A(x)} y(x))' = e ^{A(x)}a(x)y(x) + e ^{A(x)}y'(x)
\]

quindi (sempre chiaramente nell'intervallo $[a,b]$):

\[
    (e ^{A(x)}y(x))'=0
\]

Questo mi dice che:

\[
    e ^{A(x)}y(x) = costante=c \in \mathbb{R}
\]

porto dall'altra parte:

\[
    y(x) = c e ^{-A(x)}
\]

espandendo $A(x)$:

\[
    y(x) = c e ^{\int_{{}}^{{}} {a(x)} \: d{x} {}}
\]


\end{document}
